\documentclass[10pt, reqno]{amsart}
\input{../../Preambule}
\newcommand{\bxi}{ \boldx_i }
\usepackage{tikz}
\usepackage{enumitem}
\usepackage{txfonts}
\usepackage{mathrsfs}
\usepackage{fancyhdr}



\usepackage{fancyhdr}
\usetikzlibrary{calc,decorations.pathmorphing,patterns}
\usepackage{listofitems}
\tikzstyle{mynode}=[thick,draw=blue,fill=blue!20,circle,minimum size=22]
\usepackage{amsaddr}
\usepackage{mathrsfs}

\makeatletter

\pgfdeclaredecoration{penciline}{initial}{
    \state{initial}[width=+\pgfdecoratedinputsegmentremainingdistance,auto corner on length=1mm,]{
        \pgfpathcurveto%
        {% From
            \pgfqpoint{\pgfdecoratedinputsegmentremainingdistance}
                            {\pgfdecorationsegmentamplitude}
        }
        {%  Control 1
        \pgfmathrand
        \pgfpointadd{\pgfqpoint{\pgfdecoratedinputsegmentremainingdistance}{0pt}}
                        {\pgfqpoint{-\pgfdecorationsegmentaspect\pgfdecoratedinputsegmentremainingdistance}%
                                        {\pgfmathresult\pgfdecorationsegmentamplitude}
                        }
        }
        {%TO 
        \pgfpointadd{\pgfpointdecoratedinputsegmentlast}{\pgfpoint{1pt}{1pt}}
        }
    }
    \state{final}{}
}
\makeatother

\pagestyle{fancy}
%\renewcommand{\subsection{mark}[1]{\markright{#1}{}}
\fancyhead{}
\fancyfoot{} 
%\fancyhead[LE,LO]{\tiny{\thepage}}
\fancyhead[C]{\small\textsc{Économètrie: L3 MIASH, S2}}
%fancyhead[CE,CO]{\tiny{\rightmark}}
\fancyhead[L]{\small\textsc{Université de Grenoble Alpes}}
\fancyfoot[C]{\small{\thepage}}
%\fancyfoot[R]{\small \textcopyright \ \  \small\textsc
\fancyhead[R]{ \small\textsc{M. W. Urdanivia}}
%\renewcommand{\headrulewidth}{0pt}
\renewcommand{\footrulewidth}{0pt}

%\pagenumbering{roman}


\begin{document} 
\usetikzlibrary{positioning}
\usetikzlibrary{snakes}
\usetikzlibrary{calc}
\usetikzlibrary{arrows}
\usetikzlibrary{decorations.markings}
\usetikzlibrary{shapes.misc}
\usetikzlibrary{shapes}
%\tikzset{block/.style={draw, rectangle, fill=gray!20, 
%\tikzset{empty/.style={draw, rectangle, fill=none, tex
%\tikzset{line/.style={draw, -latex'}}
%\onehalfspace

%\includepdf{trame}

\begin{titlepage}
\centering
	%\includegraphics[width=0.15\textwidth]{logoUGA2020}
	{\scshape\Large \textsc{Université de Grenoble Alpes\\(L3 MIASH, S2)}\par}
	\vspace{0.5cm}
	{\Large\bfseries \scshape\Large \textsc{ÉCONOMÉTRIE}\par}
	%{\scshape\large \textsc{Extremum Estimators(1)}\par}
	%\vspace{1cm}
	\vspace{0.5cm}
	{\Large\bfseries \textsc{Travaux} \par}
    \vspace{0.5cm}   
   {\Large\bfseries \textsc{Travail 2} \par}
	\vspace{1cm}
	{(\textsc{Cette version: \today})\par}
	\vspace{1cm}
	{\large \textsc{Michal W. Urdanivia}
	\footnote{Contact:  
	\href{mailto:michal.wong-urdanivia@univ-grenoble-alpes.fr}{michal.wong-urdanivia@univ-grenoble-alpes.fr}, 
	 Université de Grenoble Alpes,  Faculté d'\'Economie, GAEL.}\par}
	 %\includegraphics[width=0.15\textwidth]{logoUGA}
	%\vfill
	%supervised by\par
	%Dr.~Mark \textsc{Brown}
%\vfill
% Bottom of the page
	
\end{titlepage}


\newpage

\tableofcontents

\newpage

\section{Application: \cite{KielMcCain1995}}
\textbf{Remarque:} pour cet exercice vous pouvez vous appuyer sur le "script" de l''exercie précédent que vous devrez donc adapter.

\medskip

On considère des données utilisées par \cite{KielMcCain1995}(fichier "KIELMC.DTA") sur des maisons vendues à Andover(MA, USA) en 1988. On considère le modèle suivant,

\begin{align*}
\log(price_i) =& \alpha + \log(dist_i)\beta + U_i
\end{align*}
où $price_i$ est le prix d'une maison $i$, et $dist_i$ sa distance par rapport à un incinérateur d'ordures. On suppose que $\Er[U_i| \log(dist_i)) = 0$, de sorte que l'estimateur des MCO est sans biais.

 \begin{enumerate}
 \item Interprétez les coefficients $\alpha$ et $\beta$. 
\item Lire les données avec un script Python. Décrivez votre
  échantillon(taille) en calculant des statistiques descriptives
  telles que les moyennes et
  écart-types des variables "price" et "dist".
 \item Estimez par MCO $\alpha$ et $\beta$. Commentez vos résultats.
 \item Pensez vous que ce modèle fournisse une mesure sans biais de l'élasticité ceteris paribus de $price_i$ par rapport à $dist_i$?(Pensez  notamment à la décision des villes quant aux placements des incinérateurs, et à la condition pour que l'estimateur des MCO soit sans biais à savoir $\Er[U_i|X_i] = 0$, autrement dit qu'en moyenne les facteurs non observés et liés à $Y_i$ ne sont pas liés avec le régresseur).
 \item Quelles autres variables affectent vraisemblablement le prix des
   maisons? Sont-elles suscéptibles d'être correlées avec la distance
   $dist_i$?.
  \item Estimez le modèle avec les variables suggérées à la question précédente et comparez vos résultats 
  par rapport au modèle de départ. 
  \item À partir de vos résultats proposez des tests de nullité des coefficients en vous appuyant sur 
  les résultats de vos estimations et les propriétés asymptotiques de l'estimateur des MCO(en supposant 
  satisfaites les conditions nécessaires à leurs validité).
 \end{enumerate}

\bibliographystyle{jpe}
\bibliography{../../Biblio.bib}
 \end{document}